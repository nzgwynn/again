\documentclass[]{sagej}
\usepackage{lmodern}
\usepackage{amssymb,amsmath}
\usepackage{ifxetex,ifluatex}
\usepackage{fixltx2e} % provides \textsubscript
\ifnum 0\ifxetex 1\fi\ifluatex 1\fi=0 % if pdftex
  \usepackage[T1]{fontenc}
  \usepackage[utf8]{inputenc}
\else % if luatex or xelatex
  \ifxetex
    \usepackage{mathspec}
  \else
    \usepackage{fontspec}
  \fi
  \defaultfontfeatures{Ligatures=TeX,Scale=MatchLowercase}
\fi
% use upquote if available, for straight quotes in verbatim environments
\IfFileExists{upquote.sty}{\usepackage{upquote}}{}
% use microtype if available
\IfFileExists{microtype.sty}{%
\usepackage{microtype}
\UseMicrotypeSet[protrusion]{basicmath} % disable protrusion for tt fonts
}{}
\usepackage[margin=1in]{geometry}
\usepackage{hyperref}
\hypersetup{unicode=true,
            pdftitle={A Priori Matching in Cluster Randomized Trials},
            pdfkeywords={cluster-randomized trials, matching},
            pdfborder={0 0 0},
            breaklinks=true}
\urlstyle{same}  % don't use monospace font for urls
\usepackage{natbib}
\bibliographystyle{plainnat}
\usepackage{longtable,booktabs}
\usepackage{graphicx,grffile}
\makeatletter
\def\maxwidth{\ifdim\Gin@nat@width>\linewidth\linewidth\else\Gin@nat@width\fi}
\def\maxheight{\ifdim\Gin@nat@height>\textheight\textheight\else\Gin@nat@height\fi}
\makeatother
% Scale images if necessary, so that they will not overflow the page
% margins by default, and it is still possible to overwrite the defaults
% using explicit options in \includegraphics[width, height, ...]{}
\setkeys{Gin}{width=\maxwidth,height=\maxheight,keepaspectratio}
\IfFileExists{parskip.sty}{%
\usepackage{parskip}
}{% else
\setlength{\parindent}{0pt}
\setlength{\parskip}{6pt plus 2pt minus 1pt}
}
\setlength{\emergencystretch}{3em}  % prevent overfull lines
\providecommand{\tightlist}{%
  \setlength{\itemsep}{0pt}\setlength{\parskip}{0pt}}
\setcounter{secnumdepth}{0}
% Redefines (sub)paragraphs to behave more like sections
\ifx\paragraph\undefined\else
\let\oldparagraph\paragraph
\renewcommand{\paragraph}[1]{\oldparagraph{#1}\mbox{}}
\fi
\ifx\subparagraph\undefined\else
\let\oldsubparagraph\subparagraph
\renewcommand{\subparagraph}[1]{\oldsubparagraph{#1}\mbox{}}
\fi

%%% Use protect on footnotes to avoid problems with footnotes in titles
\let\rmarkdownfootnote\footnote%
\def\footnote{\protect\rmarkdownfootnote}

%%% Change title format to be more compact
\usepackage{titling}

% Create subtitle command for use in maketitle
\newcommand{\subtitle}[1]{
  \posttitle{
    \begin{center}\large#1\end{center}
    }
}

\setlength{\droptitle}{-2em}
  \title{\(A\) \(Priori\) Matching in Cluster Randomized Trials}
  \pretitle{\vspace{\droptitle}\centering\huge}
  \posttitle{\par}
  \author{true}
  \preauthor{\centering\large\emph}
  \postauthor{\par}
  \predate{\centering\large\emph}
  \postdate{\par}
  \date{April 28, 2017}


\begin{document}
\maketitle
\begin{abstract}
Matching in cluster-randomized trials (CRTs) is important, but there is
no best practice. When baseline data is available, we suggest a
technique which can be used to identify the best weighting of pertinent
variables to enable more balance between treatment and control groups.
This technique involves standardizing the variables, computing the
Mahalanobis distance, use matching to find pairs, then randomizing and
rerandomizing. For each randomiziation we compute the difference in the
treatment and control arms for each variable and plot these on a
parallel-coordinates plot. Investigators can compare plots to identify
weighting for advantageous randomization schema.
\end{abstract}

\subsection{Introduction}\label{introduction}

To determine the efficacy of a treatment, individually randomized trials
(IRTs) with blinding are the ``strongest study design available''
\citep{gatsonis2017methods}. Unfortunately, cost and study design
amongst other things mean some interventions can not be randomized on an
individual level. For example, training medical doctors to improve
patient interactions would require individuals be randomized to a doctor
not of their choosing and assumes no medical emergencies treated by
other doctors for the duration of the study. To remedy this,
investigators could consider randomizing medical practices to
intervention or control groups. Trials where groups of participants are
randomized are called cluster randomized trials (CRTs). Three reasons
for conducting a CRTs are: (i) implementation occurs at the cluster
level, (ii) to avoid contamination, and (iii) to measure intervention
effects among cluster members who do not receive treatment
\citep{balzer2012match, CRTrials2009}.

Many authors discuss the importance of matching in CRTs
\citep{balzer2012match, CRTrials2009, gatsonis2017methods, murray1998design, imai2009essential, PMVsStrat, donner2007merits, klar1997merits}.
SUMMARY HERE

Despite all this information few authors discuss methodologies to
support the matching process. We suggest a method suitable for \(a\)
\(priori\) matching using baseline data. In section 2 we outline our
method, section 3 applies it to the PROTECT dataset, and section 4 is a
brief discussion.

\subsection{Methods }\label{methods}

To approach this complex topic of balancing randomization in CRTs we
suggest a new approach. Our approach involves weighting variables of
import, matching units using these weights, and randomizing many times
to obtain a distribution of possiblities when official randomization
occurs. Investigators assess these distributions to determine if
possible randomizations are sufficiently balanced, if not, weighting is
adjusted and the process begins again. The details follow.

The initial step involves prioritizing variables \((1, 2,..., m)\) from
units \((1, 2, ..., n)\) to be randomized. We have

\begin{eqnarray*}
 \overline{V_1} & = & (v_{11}, v_{12},..., v_{1n})\\
 \overline{V_2} & = & (v_{21}, v_{22},..., v_{2n})\\
 \vdots & = & \vdots\\    
 \overline{V_m} & = & (v_{m1}, v_{m2},..., v_{mn}).\\
 \end{eqnarray*}

In addition, we use \(\overline{w} = (w_{1}, w_{2},..., w_{m})\) to
weight and standardize
\((\overline{V_1}, \overline{V_2}, ..., \overline{V_m}).\) We have NOT
SURE IF THIS IS RIGHT, THINK IT'S NOT!!!!!!!! something with ij is
wrong.

\begin{eqnarray*}
v^{*}_{ii} = \frac{(v_{ii} - \frac{\sum _{j=1}^{n}v_{ij}}{n})*w_i}{sd(\overline{V_i})} 
 \end{eqnarray*}

where \(sd(\overline{V_i})\) is the standard deviation of
\(\overline{V_i}.\) We now have

\begin{eqnarray*}
 \overline{V_1^*} & = & (v_{11}^*, v_{12}^*,..., v_{1n}^*)\\
 \overline{V_2^*} & = & (v_{21}^*, v_{22}^*,..., v_{2n}^*)\\
 \vdots & = & \vdots\\    
 \overline{V_m^*} & = & (v_{m1}^*, v_{m2}^*,..., v_{mn}^*).\\
 \end{eqnarray*}

which we use to compute the Mahalanobois Distance matrix, \textbf{D}.
From here we use the \texttt{nmatch} function in the
\texttt{designmatch} package in \texttt{R} to find \(\frac{n}{2}\) pairs
if \(n\) is even. If \(n\) is odd, the remainder can be randomized to
treatment or control per the principal investigator. Without loss of
generality, we assume \(n\) is even for the remainder of this paper and
note that to include an odd \(n\) either treatment or control groups
will include one more set of priority variables.

Once the matching is completed and pairs found we return to using the
raw data, as this will be used to assess the weighting scheme. We now
have pairs
\((\overline{V}_{11}, \overline{V}_{12}), (\overline{V}_{21}, \overline{V}_{22}), ..., (\overline{V}_{\frac{n}{2}1}, \overline{V}_{\frac{n}{2}2}).\)
The first match in each pair will be randomized to either treatment or
control using the \texttt{rbinom} function in \texttt{R}. Next, we
subset \(\overline{V}_1, \overline{V}_2, ..., \overline{V}_m\) into
appropriate randomization subgroups:
\(\overline{V}_{1T}, \overline{V}_{1C}, \overline{V}_{2T}, \overline{V}_{2C},..., \overline{V}_{\frac{n}{2}T}, \overline{V}_{\frac{n}{2}C}\)
where \(\overline{V}_{iT} = (v_{i1}^T, v_{i2}^T,..., v_{in}^T),\)
similarly for \(\overline{V}_{iC}.\) Using these we find

\begin{eqnarray*}
 k_j = | \sum_{i = 1}^{\frac{n}{2}}v_{ij}^T - \sum_{i = 1}^{\frac{n}{2}}v_{ij}^C | 
\end{eqnarray*}

THE ABOVE NEEDS A BETTER ABSOLUTE VALUE SYMBOL. for
\(j = 1, 2, ..., m.\) We randomize \(N\) times and find \(k_{lj}\) the
difference in the two arms for the \(j^{th}\) priority variable for each
of the \(l = 1, 2, ..., N\) re-randomizations. To assist analysis we
draw a parallel coordinates plot where the \(j^{th}\) axis plots
\(k_{lj}\) for \(l = 1, 2, ..., N.\) If the principal investigator finds
the possible differences too large for a priority variable \(j\),
increasing \(w_j\) and re-running the above will update the matching to
attain closer matches for this variable and lessen the differences. The
penality in this process is that closer matches for variable \(j\) are
likely to imply reduced closeness in another variable, so compromises
must be made.

\subsection{Results}\label{results}

To demonstrate the usefulness of this technique we present a brief
summary of our randomization process using baseline data from the
PROTECT trial (Project PROTECT: Protecting Nursing Homes From Infections
and Hospitalization) \citep{ProtectTrial}. In this trial, the
investigators are studying whether bathing with chlorhexidine gluconate
and iodophor nasal swabs ``can reduce hospitalizations associated with
infections, antibiotic utilization, and multi-drug resistant organism
(MDRO) prevalence'' versus regular bathing. Additional training is given
to nursing homes in the treatment arm to ensure comprehension and
adherence to trial protocol.

Prior randomize baseline data was collected for ?????? on the 29????
nursing homes. With this data, investigators met to prioritize variables
into several categories: primary, secondary, tertiary, and not relevant.
For this trial, the investigators decided that percentage discharges to
hospital with infection based on primary and other diagnsoses,
percentage discharge to hospital, percentages MDRO, and percentage usage
of antibacterials started at nursing home were of primary importance. Of
secondary importance were percentage of admissions with length of stay
over 100 days, average daily census, and mean number of baths per
resident per week. If they were able to include more variables without
effecting the balance of the others they felt matching on AVERAGE
DEPENDENT late LOSS activities of daily lives, and the Centers for
Medicare and Medicaid Services (CMS) rating of eaching nursing home. To
ease understanding, our initial discussion will involve 2 variables:
percentage discharges to hospital with infection based on primary and
other diagnsoses, and average daily census.

\begin{longtable}[]{@{}lll@{}}
\caption{Abbreviations of variables used to randomize}\tabularnewline
\toprule
Primary import & Secondary import & Tetiary import\tabularnewline
\midrule
\endfirsthead
\toprule
Primary import & Secondary import & Tetiary import\tabularnewline
\midrule
\endhead
\% Inf & \% Long Stay & Late ADLs\tabularnewline
\% DC & Avg Daily Census & CMS Star\tabularnewline
MDRO & Baths/Week\tabularnewline
\% Abx &\tabularnewline
\bottomrule
\end{longtable}

Prior to randomization, investigators

\subsection{Discussion}\label{discussion}

Look how smart we are.

\bibliography{bibliography.bib}


\end{document}
